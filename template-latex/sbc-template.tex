\documentclass[12pt]{article}

\usepackage{sbc-template}
\usepackage{graphicx,url}
\usepackage[utf8]{inputenc}
\usepackage[brazil]{babel}
%\usepackage[latin1]{inputenc}  

     
\sloppy

\title{Migração de Sistemas Legados Monolíticos para Microserviços}

\author{Guilherme L. D. Villaca\inst{1}}


\address{Universidade Estadual do Oeste do Paraná (Unioeste)\\
Cascavel -- PR -- Brazil
\email{guidvillaca@gmail.com}
}

\begin{document} 

\maketitle

\begin{abstract}
  to write.
\end{abstract}
     
\begin{resumo} 
  Para Escrever.
\end{resumo}


\section{Introdução}

Segundo \cite{Bennett} podemos definir sistemas legados como vitais para a organização e que não se sabe mais como manter ou melhorá-los. Estes sistemas foram desenvolvidos com a melhor tecnologia disponível no momento de sua concepção e que com o passar do tempo tornam-se superadas, o que requer melhorias. A modernização de sistemas legados é um desafio enfrentado por muitas organizações que desejam explorar novas tecnologias de computação em nuvem para atender às necessidades de alta escalabilidade e alta disponibilidade\cite{Furda2018}. Porém, explorar novas tecnologias pode não ser a principal razão para a modernização, pois sistemas legados podem conter códigos mal escritos de difícil manutenibilidade, pobre documentação, ou ainda erros desde a elicitação de requisitos. Há ainda uma análise que deve ser feita sobre estes sistemas para a reconstrução da arquitetura de software, uma área que teve um grande avanço nos últimos anos e que conta com um grande número de técnicas e métodos que foram desenvolvidos para este fim \cite{OBrien2005}.



\section{Desenvolvimento} \label{sec:firstpage}

Desenvolver

\section{Resultados Esperados}

Resultados

\section{Cronograma de Execução}

Cronograma

%\subsection{Subsections}

%The subsection titles must be in boldface, 12pt, flush left.

\bibliographystyle{sbc}
\bibliography{sbc-template}

\end{document}
