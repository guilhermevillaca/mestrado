\documentclass[12pt]{article}

\usepackage{sbc-template}
\usepackage{graphicx,url}
\usepackage[utf8]{inputenc}
\usepackage[brazil]{babel}
     
\sloppy

\title{Migração de Sistemas Monolíticos para Microserviços}

\author{Guilherme L. D. Villaca\inst{1}}


\address{Universidade Estadual do Oeste do Paraná (Unioeste)\\
Cascavel -- PR -- Brazil
\email{guidvillaca@gmail.com}
}

\begin{document} 

\maketitle

\begin{abstract}
Migrating legacy systems through microservices has been a tendency to solve old problems faced during software development, such as scalability and maintenance. However, it is not simple to identify when these systems should migrate and what the consequences are. Unioeste Information Technology Center (NTI) has faced these problems in recent years, systems are not easily scalable and are difficult to maintain ...
\end{abstract}
     
\begin{resumo} 
Migrar sistemas legados por microserviços tem sido uma tendência para resolver velhos problemas enfrentados durante o desenvolvimento de software, como escalabilidade e manutenção. Porém não é simples identificar quando estes sistemas devem migrar, e quais as consequências disso. O Núcleo de Tecnologia da Informação(NTI) da Unioeste tem enfrentado estes problemas nos últimos anos, os sistemas não são facilmente escaláveis e são de difícil manutenção...
\end{resumo}


\section{Introdução}

O desenvolvimento de sistemas pelas organizações seguem um padrão, geralmente o desenvolvimento começa pela solução de um problema, e dessa solução novos problemas são adicionados ou agregados, formando um ou mais sistemas maiores. Com o passar dos anos esse sistema começa a gerar problemas para a equipe de desenvolvedores. Podemos citar: 
\\Manutenibilidade que é quando o sistema começa a ter problemas quanto a sua manutenção seja por bugs ou mesmo uma mal implementação, troca constante na equipe e isso acaba ocasionando défcit pois há uma clara curva de aprendizado, seja no próprio sistema ou tecnologia e conhecimento total do domínio;
\\Eescalabilidade que é quando o sistema tem picos em seu consumo e não é possível de maneira simples fazer com que ele cresça;
\\Documentação pobre ou mal feita: pode gerar problemas futuros, em relação a origem ou rastreabilidade de um requisito, por que, quando como um requisito surgiu.
\\


\cite{Dragoni2017} 

Segundo \cite{Bennett} podemos definir sistemas legados como vitais para a organização e que não se sabe mais como manter ou melhorá-los. Estes sistemas foram desenvolvidos com a melhor tecnologia disponível no momento de sua concepção e que com o passar do tempo tornaram-se superadas por tecnologias mais modernas, ágeis, versáteis e escaláveis. Outra característica destes sistemas é que eles são monolíticos, ou seja, todas as funcionalidades e códigos fonte do sistema são concentradas no mesmo projeto. A modernização de sistemas legados é um desafio enfrentado por muitas organizações que desejam explorar estas novas tecnologias para atender às necessidades de alta escalabilidade e alta disponibilidade\cite{Furda2018}. Porém, apenas explorar novas tecnologias pode não ser a principal razão para a modernização, pois sistemas legados podem conter códigos mal escritos de difícil manutenibilidade, pobre documentação, ou ainda erros desde a elicitação de requisitos.  \\
O objetivo deste trabalho é analisar os sistemas desenvolvidos pelo Núcleo de Tecnologia da Informação (NTI) da Unioeste e verificar como pode ser resolvido os problemas enfrentados pelo setor. Atualmente no NTI existem mais de 30 sistemas em produção, destes em torno de 28 são sistemas WEB, desenvolvidos com as tecnologias JAVA (vRaptor), ExtJS, Angular. Substituir estas tecnologias não é necessariamente resolver o problema principal enfrentado hoje, que é a duplicação de códigos e/ou funcionalidades. Por isso uma das alternativas é a utilização de Microserviços que segundo \cite{Carvalho2019} muitas companhias tem adotado esta tecnologia  para modernizar sistemas legados monolíticos. Ainda segundo o autor, microserviços são serviços pequenos e autônomos que trabalham juntos. Seus beneficios são reduzir esforços para manutenção e evolução, maior disponibilidade de serviços, facilidade de inovação, entrega contínua, escalabilidade facilitada de partes com mais demanda, etc.

Conforme o avanço da tecnologia e do estado da arte as organizações vão de certa forma se modernizando também, ou seja, adotando estas novas tecnologias. Algo que antes poderia não ter importância hoje se tornou um problema, os processos de desenvolvimento, deploy, testes estão cada vez mais ágeis, versáteis, variáveis. O responsável por uma empresa de TI hoje não pode mais perder tempo, se ele tem um sistema funcional hoje em um cliente ele sabe que amanhã poderá ter outro, ele não pode perder mais tempo copiando, colando, modificando seu código, pois se ele continuar fazendo isso ele será engolido por empresas que utilizando as metodologias mais avançadas do mercado. Dentre as tecnologias que hoje fazem a diferença para um melhor aproveitamento de um sistema estão TDD, deploy automatizado, containers, microserviços


falar sobre a forma de criação dos sistemas pela industria, sistemas separados porém com partes integradas
hoje na unioeste tem 28 sistemas web

dragoni define uma aplicação monolitica como um sistema composto por módulos que não são independentes da aplicação a que pertencem



\section{Desenvolvimento} \label{sec:firstpage}

Desenvolver

Titulo
Identificação autor e filicação
Resumo
palavras chave
Introdução
Desenvolvimento (uma ou mais seções, incluindo revisão bibliográfica, trabalhos relacionados e metodologia)
resultados esperados
cronograma de execução

\section{Resultados Esperados}

Resultados

\section{Cronograma de Execução}

Cronograma

%\subsection{Subsections}

%The subsection titles must be in boldface, 12pt, flush left.

\bibliographystyle{sbc}
\bibliography{sbc-template}

\end{document}
