\documentclass[12pt]{article}

\usepackage{sbc-template}
\usepackage{graphicx,url}
\usepackage[utf8]{inputenc}
\usepackage[brazil]{babel}
     
\sloppy

\title{Migração de Sistemas Monolíticos para Microserviços}

\author{Guilherme L. D. Villaca\inst{1}}


\address{Universidade Estadual do Oeste do Paraná (Unioeste)\\
Cascavel -- PR -- Brazil
\email{guidvillaca@gmail.com}
}

\begin{document} 

\maketitle

\begin{abstract}
Migrating legacy systems through microservices has been a tendency to solve old problems faced during software development, such as scalability and maintenance. However, it is not simple to identify when these systems should migrate and what the consequences are. Unioeste Information Technology Center (NTI) has faced these problems in recent years, systems are not easily scalable and are difficult to maintain ...
\end{abstract}
     
\begin{resumo} 
Migrar sistemas legados por microserviços tem sido uma tendência para resolver velhos problemas enfrentados durante o desenvolvimento de software, como escalabilidade e manutenção. Porém não é simples identificar quando estes sistemas devem migrar, e quais as consequências disso. O Núcleo de Tecnologia da Informação(NTI) da Unioeste tem enfrentado estes problemas nos últimos anos, os sistemas não são facilmente escaláveis e são de difícil manutenção...
\end{resumo}


\section{Introdução}

O desenvolvimento de sistemas pelas organizações seguem um padrão, geralmente o desenvolvimento começa pela solução especifica de um problema e a partir daí novas soluções são adicionadas, formando um ou mais sistemas maiores e complexos. Com o passar dos anos esse sistema começa a gerar problemas para a equipe de desenvolvedores. Podemos citar: 
\\Manutenibilidade que é quando o sistema começa a ter problemas quanto a sua manutenção seja por bugs ou mesmo uma mal implementação, troca constante na equipe e isso acaba ocasionando défcit pois há uma clara curva de aprendizado, seja no próprio sistema ou tecnologia e conhecimento total do domínio;
\\Eescalabilidade que é quando o sistema tem picos em seu consumo e não é possível de maneira simples fazer com que ele cresça;
\\Documentação pobre ou mal feita: pode gerar problemas futuros, em relação a origem ou rastreabilidade de um requisito, por que, quando como um requisito surgiu.

falar que estes sistemas desenvolvidos sao sistemas monoliticos \cite{Dragoni2017}

falar da origem SOA e que a partir daí surgiu microserviços

Este cenário com estes problemas relatados podem ser resolvidos de diversas maneiras, neste trabalho iremos propor uma solução baseada em microserviços, a migração de um sistema monolitico tem se tornado bastante comum na indústria, segundo \cite{Carvalho2019} alguns desenvolvedores até atribuem a mudança pelo hype ou seja, por esta abordagem estar na moda e não querer ficar pra trás.


O objetivo deste trabalho é analisar os sistemas desenvolvidos pelo Núcleo de Tecnologia da Informação (NTI) da Unioeste e verificar como pode ser resolvido os problemas enfrentados pelo setor. Atualmente no NTI existem mais de 30 sistemas em produção, destes em torno de 28 são sistemas WEB, 


dragoni define uma aplicação monolitica como um sistema composto por módulos que não são independentes da aplicação a que pertencem



\section{Desenvolvimento} \label{sec:firstpage}

Desenvolver

Titulo
Identificação autor e filicação
Resumo
palavras chave
Introdução
Desenvolvimento (uma ou mais seções, incluindo revisão bibliográfica, trabalhos relacionados e metodologia)
resultados esperados
cronograma de execução

\section{Resultados Esperados}

Resultados

\section{Cronograma de Execução}

Cronograma

%\subsection{Subsections}

%The subsection titles must be in boldface, 12pt, flush left.

\bibliographystyle{sbc}
\bibliography{sbc-template}

\end{document}
