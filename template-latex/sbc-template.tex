\documentclass[12pt]{article}

\usepackage{sbc-template}
\usepackage{graphicx,url}
\usepackage[utf8]{inputenc}
\usepackage[brazil]{babel}
     
\sloppy

\title{Migração de Sistemas Monolíticos para Microserviços}

\author{Guilherme L. D. Villaca\inst{1}}


\address{Universidade Estadual do Oeste do Paraná (Unioeste)\\
Cascavel -- PR -- Brazil
\email{guidvillaca@gmail.com}
}

\begin{document} 

\maketitle

\begin{abstract}
Migrating legacy systems through microservices has been a tendency to solve old problems faced during software development, such as scalability and maintenance. However, it is not simple to identify when these systems should migrate and what the consequences are. Unioeste Information Technology Center (NTI) has faced these problems in recent years, systems are not easily scalable and are difficult to maintain ...
\end{abstract}
     
\begin{resumo} 
Migrar sistemas legados por microserviços tem sido uma tendência para resolver velhos problemas enfrentados durante o desenvolvimento de software, como escalabilidade e manutenção. Porém não é simples identificar quando estes sistemas devem migrar, e quais as consequências disso. O Núcleo de Tecnologia da Informação(NTI) da Unioeste tem enfrentado estes problemas nos últimos anos, os sistemas não são facilmente escaláveis e são de difícil manutenção...
\end{resumo}


\section{Introdução}

%maintainability ingles para manutenibilidade

A tecnologia de software baseado em reúso é um processo de design de software que pode resultar em redução de tempo e custos de desenvolvimento além de aumentar a flexibilidade, manutenibilidade e confiabilidade do software. A principal razão para pensar em reúso é evitar trabalhos repetidos no desenvolvimento de software e usar do conhecimento e experiência adquiridos previamente e concentrar os esforços em partes mais críticas da aplicação. Seu foco é não necessitar mais desenvolver um software do zero \cite{Yang}.  
\\
Reengenharia é uma forma para atingir o reuso de software e para entender os conceitos ocultos no domínio da aplicação. Seu objetivo é obter e manter o conhecimento incluído nos sistemas legados, usando-o como base para a continuidade e evolução estruturada do sistema. O código legado tem lógicas programadas, decisões de projeto, requisitos de usuários e regras de negócio que podem ser recuperadas e reconstruídas sem perder a semântica \cite{Garcia2004a}
\\
O desenvolvimento de sistemas pelas organizações segue um padrão, onde o desenvolvimento começa pela solução especifica de um problema e a partir daí novas soluções são adicionadas, formando um ou mais sistemas maiores e complexos. Com o passar dos anos esses sistemas começam a gerar problemas para a equipe de desenvolvedores. Podemos citar: 

%essa parte entra no desenvolvimento
\\Manutenibilidade: Que é a capacidade de um produto de software ser modificado. Se um software não for construído com capacidade para lidar com a demanda de constantes mudanças, certamente irá consumir mais recursos e tornar a manutenção uma tarefa tediosa \cite{Velmourougan2014};
\\Escalabilidade: É a capacidade de um software para acomodar crescimento em termos de tamanho e complexidade do problema e em um segundo aspecto está relacionado a uma medida de eficácia quando usada em um contexto maior em escopo e complexidade do que o contexto para o qual foi originalmente projetado. \cite{Ibrahim2009}
\\Confiabilidade: Capacidade do sistema estar livre de falhas \cite{pan1999}. Não importa o tamanho do erro ou o quanto ele afeta o sistema, corrigir um erro de forma equivocada pode gerar novos problemas.
\\Qualidade
\\Documentação pobre ou mal feita: pode gerar problemas futuros, em relação a origem ou rastreabilidade de um requisito, por que, quando como um requisito surgiu.

O jeito tradicional de desenvolvimento resulta em sistemas monoliticos que é um sistema composto por módulos que não são independentes da aplicação a que pertencem \cite{Dragoni}

Algumas abordagens foram propostas para resolver estes problemas clássicos dos sistemas monolíticos, como o SOA (Service Oriented Architeture) que foi algo surgido na academia, porém sua implementação não era tão simples e acabou não sendo adotado pela indústria. Alguns anos atrás surgiu algo similar ao SOA, os microserviços porém dessa vez proposto pela indústria

Este cenário com estes problemas relatados podem ser resolvidos de diversas maneiras, neste trabalho iremos propor uma solução baseada em microserviços, a migração de um sistema monolitico tem se tornado bastante comum na indústria, segundo \cite{Carvalho} alguns desenvolvedores até atribuem a mudança pelo hype ou seja, por esta abordagem estar na moda e não querer ficar pra trás.


O objetivo deste trabalho é analisar os sistemas desenvolvidos pelo Núcleo de Tecnologia da Informação (NTI) da Unioeste e verificar como pode ser resolvido os problemas enfrentados pelo setor. Atualmente no NTI existem mais de 30 sistemas em produção, destes em torno de 28 são sistemas WEB, 


\section{Desenvolvimento} \label{sec:firstpage}

Desenvolver

Titulo
Identificação autor e filicação
Resumo
palavras chave
Introdução
Desenvolvimento (uma ou mais seções, incluindo revisão bibliográfica, trabalhos relacionados e metodologia)
resultados esperados
cronograma de execução

\section{Resultados Esperados}

Resultados

\section{Cronograma de Execução}

Cronograma

%\subsection{Subsections}

%The subsection titles must be in boldface, 12pt, flush left.

\bibliographystyle{sbc}
\bibliography{sbc-template}

\end{document}
