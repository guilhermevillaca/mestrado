\documentclass[12pt]{article}

\usepackage{sbc-template}
\usepackage{graphicx,url}
\usepackage[utf8]{inputenc}
\usepackage[brazil]{babel}
     
\sloppy

\title{Migração de Sistemas Monolíticos para Microserviços}

\author{Guilherme L. D. Villaca\inst{1}}


\address{Universidade Estadual do Oeste do Paraná (Unioeste)\\
Cascavel -- PR -- Brazil
\email{guidvillaca@gmail.com}
}

\begin{document} 

\maketitle

\begin{abstract}
Migrating monolithic systems to microservices has been a way for developers to solve old problems faced in the software lifecycle, such as scalability and maintainability. However, it is not simple to identify when these systems should migrate, and what the consequences are. In theory, applying microservices brings a lot of benefits compared to monolithic systems, especially when it is previously planned and starts development from scratch. In practice, issues like sharing a database, team relocation, infrastructure investment, developer team re-education are no easy task.
\end{abstract}
     
\begin{resumo} 
Migrar sistemas monolíticos para microserviços tem sido uma maneira para desenvolvedores resolverem velhos problemas enfrentados durante o ciclo de vida de um software, como escalabilidade e manutenibilidade. Porém não é simples identificar quando estes sistemas devem migrar, e quais as consequências disso. Na teoria, aplicar microserviços traz uma série de benefícios se comparado aos sistemas monolíticos, ainda mais quando se planeja previamente e inicia o desenvolvimento do zero. Na prática, questões como dividir uma base de dados, remanejamento de equipe, investimento em infra-estrutura, reeducação da equipe de desenvolvedores não é tarefa fácil.
\end{resumo}

{\bf Palavras-chave:} sistemas monolíticos, microserviços, reúso de software, reengenharia, migração, banco de dados, manutenibilidade, escalabilidade.


\section{Introdução}

%maintainability ingles para manutenibilidade

A técnica de desenvolvimento de software baseado em reúso é um processo de design que pode resultar em redução de tempo e custos de desenvolvimento além de aumentar a flexibilidade, manutenibilidade e confiabilidade do software. A principal razão para pensar em reúso é evitar trabalhos repetidos no desenvolvimento de software e usar do conhecimento e experiência adquiridos previamente e concentrar os esforços em partes mais críticas da aplicação. Seu foco é não necessitar mais desenvolver um software do zero \cite{Yang}.  
\\
Reengenharia pode ser uma forma para atingir o reuso de software e para entender os conceitos ocultos no domínio da aplicação. Seu objetivo é obter e manter o conhecimento incluído nos sistemas legados, usando-o como base para a continuidade e evolução estruturada do sistema. O código legado tem lógicas programadas, decisões de projeto, requisitos de usuários e regras de negócio que podem ser recuperadas e reconstruídas sem perder a semântica \cite{Garcia2004a}
\\
Os sistemas legados são tradicionalmente construídos de forma monolítica, ou seja, um sistema composto por módulos que não são independentes da aplicação a que pertencem \cite{Dragoni2017}. Para resolver as limitações dos sistemas monolíticos, os Microserviços estão sendo amplamente utilizados pela indústria por ser uma técnica que estimula uma melhor modularização e gestão através de serviços menores e autônomos \cite{Carvalho2019}. Ter um sistema mais adaptável a mudanças é a melhor estratégia para modernizar sistemas legados \cite{Kamimura2019a}.
\\
O objetivo deste trabalho é verificar como seria a migração de sistemas monolíticos para microserviços em um ambiente com múltiplos sistemas interligados. Em um cenário como esse como deve ocorrer a migração? De uma forma mais gradual e a longo prazo? Somente as pequenas partes interligadas poderiam se tornar microserviços? Apenas novas funcionalidades devem ser desenvolvidas com a nova arquitetura? Desenvolver tudo do zero? Apesar de existir no estado da arte guias e processos para migração, não há uma definição sobre como proceder em um cenário com múltiplos sistemas. 

\section{Desenvolvimento} \label{sec:firstpage}

%%revisão bibliográfica
\subsection{Revisão Bibliográfica}

O desenvolvimento de sistemas geralmente segue um padrão onde o desenvolvimento começa pela solução específica de um problema e a partir daí novas soluções podem ser adicionadas, formando um ou mais sistemas maiores e complexos. Com o passar dos anos estes sistemas podem gerar diversos problemas para a equipe de desenvolvedores. Alguns dos principais problemas são:
%essa parte entra no desenvolvimento
\begin{itemize}
	\item Manutenibilidade: Que é a capacidade de um software ser modificado. Se um software não for construído com capacidade para lidar com a demanda de constantes mudanças, certamente irá consumir mais recursos e tornar a manutenção uma tarefa tediosa \cite{Velmourougan2014}.
	\item Escalabilidade: É a capacidade de um software para acomodar crescimento em termos de tamanho e complexidade do problema e em um segundo aspecto está relacionado a uma medida de eficácia quando usada em um contexto maior em escopo e complexidade do que o contexto para o qual foi originalmente projetado \cite{Ibrahim2009}.
	\item Confiabilidade: Eficácia do sistema quanto a estar livre de falhas por um período de tempo. É difícil atingir a confiabilidade de um software, pois quanto mais complexo, menos confiável ele se torna.\cite{pan1999}.
	\item Qualidade: Garantia de que o software atenda a todos os requisitos e a capacidade do software de atender às necessidades dos usuários.
\end{itemize}

%a migração para microserviços pode trazer alguns beneficios
%pontos positivos, melhorar manutenibilidade, uso de novas tecnologias, delegação de responsabilidades, custos a longo prazo

\subsubsection{Benefícios da Adoção de Microserviços}
Além de resolver os problemas citados acima, a adoção de microsserviços trás alguns outros benefícios como \textbf{delegação de responsabilidades} já que microserviços podem ser desenvolvidos por times de desenvolvedores diferentes e independentes, reduzindo sobrecarga de comunicação e necessidade de coordenação entre as equipes \cite{Taibi}; 
\\
\textbf{Diminuição de Custos}: segundo \cite{Taibi}, apesar dos custos iniciais do desenvolvimento com microserviços serem maiores, a longo prazo esse custo é menor, se comparado com sistemas monolíticos, devido a redução, principalmente, de manutenibilidade; 
\\
\textbf{Novas tecnologias} pois com serviços independentes, não importa como esse serviço é implementado, portanto é possível escolher diferentes tecnologias que se adequem às necessidades especificas daquela funcionalidade e isso também se aplica a banco de dados. Não há necessidade de se prender a uma tecnologia \cite{Richter2017a}. 
%por no slide tolerancia a falhas como beneficios=confiabilidade.	


%banco de dados é um ponto negativo, resistencia a mudanças da equipe, custos iniciais maiores
\subsubsection{Riscos da Adoção de Microserviços}
A utilização de microsserviços também trás alguns riscos consideráveis como: lidar com a base de dados pode ser um desafio principalmente em ambientes com serviços críticos. Em um típico sistema monolítico, a base de dados segue esta arquitetura. Quando o foco é microserviços, cada serviço deve ser independente, responsável pelos seus próprios dados para que qualquer mudança não afete o sistema todo \cite{Kholy2019}. Mesmo que uma entidade seja comum a dois ou mais microserviços, ainda assim pode haver diferenças entre elas, o que apoia a necessidade de separação. 
\\É comum haver uma resistência em equipes de desenvolvimento, isso ocorre devido a dificuldade em aceitar mudanças, seja de arquitetura e tecnologias, ou ainda a apego ao software desenvolvido \cite{Taibi}. 
\\Outro risco se refere a custos iniciais maiores, mesmo que a longo prazo seja vantajoso, em um primeiro momento o investimento necessário pode ser uma barreira, pois não só é preciso investimento em infra-estrutura, como também pode ser essencial uma reestruturação na equipe de desenvolvimento \cite{Taibi}.

\subsubsection{Estratégias para a Migração}

Existem algumas estratégias para a migração de um sistema monolítico para microserviços. Segundo \cite{Taibi} a migração pode ser feita a partir de novas funcionalidades do sistema, ou após a identificação e decomposição do sistema existente, desenvolver cada microserviço praticamente do zero, pois é possível reutilizar códigos e regras de negócio. A abordagem selecionada para este trabalho será a decomposição do sistema existente e desenvolvimento dos microserviços. Uma das dificuldades é manter e testar os dois ambientes (monolítico e microserviço) em paralelo. Uma forma de resolver este problema é utilizar a \textit{feature toggle} de Martin Fowler \cite{Carvalho2019} que alterna entre monolítico e microserviço conforme a necessidade. Dessa forma é possível testar com mais tranquilidade a nova arquitetura. 


\subsection{Trabalhos Relacionados}
%trabalhos relacionados
Existem alguns trabalhos que focaram na migração de sistemas monolíticos para microserviços.
\\Na dissertação "Um Guia Para Apoiar a Migração de Sistemas de Software Legados Para Arquitetura Baseada em Microseviços"\hspace{0.05cm} o autor propõe um roteiro com etapas a se seguir para migração de sistemas legados, focando em identificar funcionalidades candidatas a serem modularizadas no contexto de microserviços. O autor também faz a aplicação em um teste real deste roteiro.
\\No trabalho "Modernização de Sistemas Legados Para Disponibilização em Dispositivos Móveis com Arquitetura Baseada em Microservices" o autor propõe uma migração para atender a distribuição e independência de um sistema legado para disponibilização em dispositivos móveis. Para o estudo de caso foi utilizado um sistema acadêmico.
No artigo "Microservices migration partterns" o autor \cite{Balalaie2018} trás uma série de padrões para a migração. O trabalho parte da premissa de que é necessário escolher fatores mais importantes para a migração, ou seja, o que se deseja atingir com a migração, melhoria de escalabilidade, tolerância a falhas, dinamicidade entre outras. Após a escolha dos fatores, onde cada fator corresponde a um ou mais padrões, tem-se uma combinação e uma sequência de etapas para seguir. Este trabalho também foi aplicado para a migração de um sistema e validado em outras três empresas.
\\Foi identificado outros trabalhos similares, porém todos eles aplicados a um sistema específico. Neste contexto a migração é mais simples comparada ao cenário que é o foco deste trabalho, onde vários sistemas devem ser afetados pela arquitetura de microserviços.
%metodologia
\subsection{Metodologia}
De acordo com \cite{PERRUNESON2012} em uma pesquisa de estudo de caso, nós escolhemos estudar um fenômeno que ocorre em um ambiente da vida real. Nossa "unidade de análise"\hspace{0.05cm} pode ser algum aspecto de um projeto de engenharia de software, uma metodologia de engenharia de software e seu uso dentro de uma organização, o todo ou uma parte específica de um novo projeto em desenvolvimento ou manutenção de um projeto em andamento.
\\Hoje o Núcleo de Tecnologia da Informação (NTI) da UNIOESTE vivencia um cenário com vários sistemas monolíticos que sofrem por problemas de manutenibilidade, confiabilidade entre outros. O objetivo é analisar a arquitetura existente e propor uma nova arquitetura que atenda aos principais conceitos de Microserviços.
\\Inicialmente fizemos pesquisas em portais de artigos e periódicos para entender como estão os estudos atuais na área, buscamos pelos assuntos: \textit{microservices architecture}, \textit{software reliability}, \textit{reengineering legacy system}, \textit{monolithic system}, \textit{software reuse}, \textit{software maintanability}, \textit{quality of software}.
 
\section{Resultados Esperados}

Após os estudar e analisar a arquitetura existente cenário alvo deste estudo de caso, espera-se a migração de uma aplicação, em todo ou em parte, para a arquitetura de Microserviços. Em um segundo momento, analisar os efeitos da migração, benefícios e dificuldades encontradas. Um dos riscos citados é a separação da base de dados, portanto pode ser a parte mais crítica da migração.

\section{Cronograma de Execução}

\begin{enumerate}
	\item Levantar um histórico de todos os sistemas existentes no Núcleo de Tecnologia da Informação (NTI) da UNIOESTE, para identificar a arquitetura existente;
	\item Identificar dependências e gerar lista de interdependências entre os sistemas;
	\item Definição da nova arquitetura;
	\item Decomposição da lógica, definir protocolos de comunicação, frameworks e ferramentas, definição de plano B;
	\item Decomposição da base de dados;
	\item Implementação de forma gradual, devido ao sisteme monolítico estar em funcionamento paralelamente;
	\item Análise de possíveis problemas e melhoria do processo.
	
\end{enumerate}

%\subsection{Subsections}

%The subsection titles must be in boldface, 12pt, flush left.

\bibliographystyle{sbc}
\bibliography{sbc-template}

\end{document}
